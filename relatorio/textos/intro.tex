\section{Introdução} \label{sec:intro}

Neste trabalho foram implementadas técnicas de limiarização de imagens monocromáricas. Esses processamento são comumente utilizados para destacar objetos na imagem, separando-os do fundo. Essa técnicas marcam o objeto encontrado como preto na imagem e o fundo como branco, por isso elas também são chamadas de técnicas de binarização.

Os oitos métodos de limiarização do trabalho envolvem limites em em toda a imagem, no método global, ou em apenas uma vizinhança do pixel, nos métodos locais. As técnicas foram comparadas em sete imagens distintas, apresentadas na \cref{sec:imgbase}, e vários parâmetros de controle de cada método, descritos junto com o método na \cref{sec:resultados}.
