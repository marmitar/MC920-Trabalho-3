\section{Método da Mediana}

O último método seleciona o limiar como a mediana da vizinhança. Isso faz com que a probabilidade do pixel de ser maior ou menor que o limiar fique em torno de $50\%$, gerando muito ruído na imagem. Podemos ver esse efeito em objetos simples, de apenas um cor, como na \cref{fig:mediana:wedge}.

Em objetos detalhados (\cref{fig:mediana:peppers}), a técina acaba mantendo o aspecto da figura original, em vez de enfatizar os objetos da figura.

\begin{figure}[H]
    \centering
    \metodo{wedge}{mediana}{}{$r = 50$}{42}%
    \metodo{peppers}{mediana}{}{$r = 60$}{50}

    \caption{Exemplos de aplicação do método \texttt{mediana}.}
    \label{fig:mediana}
\end{figure}
