\subsection{Método Global}

\begin{figure}[H]
    \centering
    \metodo[wide]{retina}{global}{}{$T = 128$}{15}\\[8pt]
    \metodo{wedge}{global}{100}{$T = 100$}{72}%
    \metodo{wedge}{global}{110}{$T = 110$}{49}

    \caption{Exemplos de aplicação do método \texttt{global}.}
    \label{fig:global}
\end{figure}

Esse é o método mais simples, definindo um limiar fixo $T$ para toda a imagem. Ele funciona razoavelmente bem para imagens com iluminação regular, como pode ser visto na \cref{fig:global:retina}.

Por outro lado, figuras com iluminação mais natural ou irregular podem ter uma grande variação da intensidade dos pixels ao longo do objeto. Isso faz com a separação do objeto de forma global seja praticamente impossível, levando a problemas como os das figuras \ref{fig:global:wedge100} e \ref{fig:global:wedge110}.