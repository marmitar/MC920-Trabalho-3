\documentclass[assignment = 3]{homework}

\usepackage{caption, subcaption, pdfpages, float}
\usepackage{graphics, wrapfig, pgf, graphicx}
\usepackage{enumitem}
\graphicspath{{../resultados/}}


% pacotes para importar código
\usepackage{caption, booktabs}
\usepackage[section, newfloat]{minted}
\definecolor{sepia}{RGB}{252,246,226}
\setminted{
    bgcolor = sepia,
    style   = pastie,
    frame   = leftline,
    autogobble,
    samepage,
    python3,
}
\setmintedinline{
    bgcolor={}
}

% ambientes de códigos de Python
\newmintedfile[pyinclude]{python3}{}
\newmintinline[pyline]{python3}{}
\newcommand{\pyref}[2]{\href{#1}{\texttt{#2}}}

% \SetupFloatingEnvironment{listing}{name=Código}
% \captionsetup[listing]{position=below,skip=-1pt}

\usepackage{csquotes}
\usepackage[style=verbose-ibid,autocite=footnote,notetype=foot+end,backend=biber]{biblatex}
\addbibresource{referencias.bib}
\usepackage[section]{placeins}

\usepackage[hidelinks]{hyperref}
\usepackage[noabbrev, nameinlink, brazilian]{cleveref}
\hypersetup{
    pdftitle  = {MC920 - Trabalho 3 - 187679},
    pdfauthor = {Tiago de Paula}
}

\newcommand{\textref}[2]{
    \hyperref[#2]{#1 \ref*{#2}}
}

\usepackage{import, multirow}
\usepackage{tikz}
\usetikzlibrary{matrix}
\usetikzlibrary{positioning}

\newenvironment{kmatrix}[1][1.3cm]{
    \begin{tikzpicture}[node distance=0cm]
        \tikzset{square matrix/.style={
                matrix of nodes,
                column sep=-\pgflinewidth, row sep=-\pgflinewidth,
                nodes={draw,
                    minimum height=#1,
                    anchor=center,
                    text width=#1,
                    align=center,
                    inner sep=0pt
                },
            },
            square matrix/.default=#1
        }
}{
    \end{tikzpicture}%
}

\newcommand*{\Scale}[2][4]{\scalebox{#1}{\ensuremath{#2}}}%

\newcommand{\red}[1]{\textcolor{red}{\textbf{#1}}}

\date{1 de dezembro de 2020}

\begin{document}

    \pagestyle{main}

    \section{Introdução} \label{sec:intro}

Neste trabalho foram implementadas técnicas de limiarização de imagens monocromáricas. Esses processamento são comumente utilizados para destacar objetos na imagem, separando-os do fundo. Essa técnicas marcam o objeto encontrado como preto na imagem e o fundo como branco, por isso elas também são chamadas de técnicas de binarização.

Os oitos métodos de limiarização do trabalho envolvem limites em em toda a imagem, no método global, ou em apenas uma vizinhança do pixel, nos métodos locais. As técnicas foram comparadas em sete imagens distintas, apresentadas na \cref{sec:imgbase}, e vários parâmetros de controle de cada método, descritos junto com o método na \cref{sec:resultados}.


    \section{O Programa}

O corpo do programa foi desenvolvido em Python 3.6+, utilizando a biblioteca padrão em conjunto com as bibliotecas OpenCV, para leitura e escrita das imagens, NumPy e SciPy, para as técnicas de limiarização local.

Em especial, a limiarização local foi implementada com a função \mintinline{python3}{generic_filter} \autocite{ref:genericfilter} do SciPy. Para acelerar um pouco o processamento, as funções de cálculo do limiar que são passadas como argumento da \mintinline{python3}{generic_filter} foram implementadas em C (padrão GNU11) e requerem o compilador GCC 4.7+ na primeira execução do programa.

\subsection{Código Fonte}

    O código fonte foi separado nos seguintes arquivos, dentro da pasta \texttt{limiarizacao}:

    \begin{description}
        \item[\_\_main\_\_.py] Processamento os comandos e as opções da linha de comando.

        \item[metodos] Pacote interno com as operações de limiarização.

        \begin{description}
            \item[metodos/\_\_init\_\_.py] Classes e objetos para as técnicas de limiarização e tratamento dos parâmetros de cada técnica, pelos argumentos da linha de comando.

            \item[metodos/locais.py] Wrapper para a chamada das funções em C.

            \item[metodos/locais.c] Funções de cálculo do limiar da vizinhança do pixel.

            \item[metodos/ops.c] Operações de mínimo, máximo, média, desvio padrão e mediana.
        \end{description}

        \item[inout.py] Tratamento de entrada e saída do programa.

        \item[tipos.py] Tipos para checagem estática com MyPy.
    \end{description}

    Além disso, existem outros arquivos junto com o código fonte. A pasta \texttt{imagens} contém as figuras base utilizadas neste relatório, apresentadas na \red{SECIMG}. \red{FOI ??} disponibilizado também um \textit{script} em \texttt{bash}, \texttt{run.sh}, que realiza todos os processamentos apresentados no relatório \red{RESULTADOS?}.

\subsection{Execução} \label{sec:execucao}

    A execução deve ser feita através do interpretador de Python, com as seguintes entradas obrigatórias: o caminho para a imagem PGM que será processada e o método de limiarização a ser aplicado. Ao final da execução, a imagem resultante será exibida na tela.

    Os argumentos opicionais podem ser vistos com \mintinline{bash}{$ python3 limiarizacao --help}. A primeira opção é \mintinline{text}{--output}, ou \mintinline{text}{-o}, que salva o resultado em um arquivo PNG ou PGM em vez de exibir na tela. Se é desejável tanto a exibição da imagem quanto o salvamento no arquivo, o argumento \mintinline{text}{--force-show} ou \mintinline{text}{-f} pode ser usado.

    O método de limiarização deve vir após essas opções e pode ser uma das seguintes chaves: \mintinline{bash}{global}, \mintinline{bash}{bernsen}, \mintinline{bash}{niblack}, \mintinline{bash}{sauvola}, \mintinline{bash}{phansalkar}, \mintinline{bash}{contraste}, \mintinline{bash}{media} ou \mintinline{bash}{mediana}. Os parâmetros de cada método pode ser visto com \mintinline{bash}{$ python3 limiarizacao METODO --help}, onde \mintinline{bash}{METODO} é a técnica desejada.

    Por exemplo, o comando abaixo apresenta executa o método de Bernsen com vizinhança de raio 6 em uma imagem \mintinline{bash}{entrada.pgm}, salvando depois o resultado em \mintinline{bash}{saida.png}.

    \begin{minted}{bash}
        $ python3 limiarizacao entrada.pgm -o saida.png bernsen -r 6
    \end{minted}


    \section{Implementação} \label{sec:impl}

\subsection{Técnica de Meios-Tons com Difusão de Erro}

    Essa técnica de pontilhado se baseia em alterar cada pixel com escala de 8 bits para escala de 2 bits, levando para o seu nível de intensidade mais próximo. No entanto, ao longo do processo, o erro é calculado em relação ao novo pixel, que é então distribuído na sua vizinhaça, influenciando as aplicações nos pixels seguintes.

    Assim, quando um pixel resulta em valor muito distante do original, seus vizinhos são reduzidos ou incrementados, aumentando a chance de que eles sejam transformados para outro nível. Isso faz com que a vizinhaça mantenha um pouco mais da distribuição local de intensidade, deixando a imagem resultante mais similar à original, considerando o nosso sistema visual.

    A distribuição de erros pode ser feita de várias formas, como pode ser visto na \cref{sec:distribuicoes}.

\subsection{Formas de Varredura} \label{sec:varredura}

    A técnica de pontilhado com distribuição de erros altera a imagem enquanto é aplicada, o que faz com que o caminhos diferentes levam à resultados distintos. Por isso, o pontilhado foi implementado seguindo quatro formas de varredura, apresentadas na \cref{fig:varredura}. Apesar disso, as varreduras que serão mais discutidas aqui serão a unidirecional (\ref{fig:varredura:unidirecional}) e a alternada (\ref{fig:varredura:alternada}).

    % \begin{figure}[H]
    %     \centering
    %     \input{figuras/varredura}

    %     \caption{Argumentos válidos para \mintinline{bash}{--varredura} ou \mintinline{bash}{-v}.}
    %     \label{fig:varredura}
    % \end{figure}

    Para a varredura alternada, a distribuição é aplicada em duas orientações, normal e invertida horizontalmente, enquanto para as curvas em espiral e de Hilbert a distribuição é considerada em quatro orientações, seguindo as rotações sucessivas de 90\textdegree{} na matriz.

\subsection{Distribuições de Erro} \label{sec:distribuicoes}

    A distribuições são escolhidas na linha de comando pelo nome de um de seus idelizadores ou de todos juntos, separados por "\texttt{\_}". Assim, as seguintes opções são equivalentes:

    \begin{minted}{bash}
        $ python3 main.py imagens/peppers.png -d jarvis
        # ou
        $ python3 main.py imagens/peppers.png -d Judice
        # ou
        $ python3 main.py imagens/peppers.png -d JARVIS_judice_Ninke
    \end{minted}

    As distribuições estão apresentadas na \cref{fig:distribuicoes} abaixo.

    % \begin{figure}[H]
    %     \centering
    %     \input{figuras/dists}
    %     \caption{Distribuições de erro aplicadas neste trabalho.}
    %     \label{fig:distribuicoes}
    % \end{figure}


    \section{Resultados}

\captionsetup{justification=centering}
\newlength{\figwidth}
\newcommand{\figratio}{0.9}
\newcommand{\metodo}[6][]{%
    \ifblank{#1}{%
        \begin{subfigure}{0.45\textwidth}
            \centering\setlength{\figwidth}{\figratio\textwidth}
            \includegraphics[width=\figwidth]{#3/#2#4.png}
            \caption{~\texttt{#2.pgm} com #5.\\ $#6\%$ de pixels pretos.}
            \label{fig:#3:#2#4}
        \end{subfigure}%
    }{%
        \begin{subfigure}{0.9\textwidth}
            \centering\setlength{\figwidth}{\figratio\textwidth}
            \includegraphics[width=0.5\figwidth]{#3/#2#4.png}
            \caption{~\texttt{#2.pgm} com #5. $#6\%$ de pixels pretos.}
            \label{fig:#3:#2#4}
        \end{subfigure}%
    }
}

\subsection{Método Global}

\begin{figure}[H]
    \centering
    \metodo[wide]{retina}{global}{}{$T = 128$}{15}\\[8pt]
    \metodo{wedge}{global}{100}{$T = 100$}{72}%
    \metodo{wedge}{global}{110}{$T = 110$}{49}

    \caption{Exemplos de aplicação do método \texttt{global}.}
    \label{fig:global}
\end{figure}

Esse é o método mais simples, definindo um limiar fixo $T$ para toda a imagem. Ele funciona razoavelmente bem para imagens com iluminação regular, como pode ser visto na \cref{fig:global:retina}.

Por outro lado, figuras com iluminação mais natural ou irregular podem ter uma grande variação da intensidade dos pixels ao longo do objeto. Isso faz com a separação do objeto de forma global seja praticamente impossível, levando a problemas como os das figuras \ref{fig:global:wedge100} e \ref{fig:global:wedge110}.
\subsection{Método de Bernsen}

Para o método de Bernsen, o limiar é calculado como a média da menor ($z_\text{min}$) e da maior ($z_\text{máx}$) intensidade na vizinhança de raio $r$, ou seja, $T(x, y) = \left(z_\text{min} + z_\text{máx}\right) / 2$. Esse método funciona melhor com imagens com alta variação de intensidade como as figuras \ref{fig:bernsen:baboon} e \ref{fig:bernsen:peppers}.

Para regiões de baixa variação de intensidade, como acontece nas imagens \texttt{wedge.pgm} e \texttt{sonnet.pgm}, o método de Bernsen gera artefatos inesperados, seguindo o gradiente da imagem. No entanto, o método funciona bem para evidenciar bordas dos objetos, como pode ser visto em \ref{fig:bernsen:wedge} e \ref{fig:bernsen:sonnet}.

\begin{figure}[H]
    \centering
    \metodo{baboon}{bernsen}{}{$r = 50$}{57}%
    \metodo{peppers}{bernsen}{}{$r = 50$}{53}\\[8pt]
    \metodo{wedge}{bernsen}{}{$r = 20$}{49}%
    \metodo{sonnet}{bernsen}{}{$r = 20$}{61}

    \caption{Exemplos de aplicação do método \texttt{bernsen}.}
    \label{fig:bernsen}
\end{figure}
\subsection{Método de Niblack}

O limiar desse método é encontrado pela média da vizinhança $\mu(x, y)$, com mais uma parcela $k$ do desvio padrão $\sigma(x, t)$, de forma que $T(x, y) = \mu(x, y) + k \sigma(x, y)$. Isso faz com o que o modelo seja mais robusto para reduzir o ruído, especialmente em regiões de baixa variação de intensidade. Podemos ver isso nas imagens \cref{fig:niblack:wedge} e \cref{fig:niblack:sonnet50}.

Por outro lado, o método pode acabar reduzindo os detalhes, no caso da vizinhança sem muito grande. Isso fica visível na diferença das letras do soneto da \cref{fig:niblack:sonnet50} e da \ref{fig:niblack:sonnet20}. Entretanto, para imagens com detalhes maiores, como a \cref{fig:niblack:fiducial}, o método de Niblack pode funcionar bem.

\begin{figure}[H]
    \centering
    \metodo{wedge}{niblack}{}{$r = 50$ e $k = -0.3$}{71}%
    \metodo{fiducial}{niblack}{}{$r = 50$ e $k = -0.5$}{67}\\[8pt]
    \metodo{sonnet}{niblack}{50}{$r = 50$ e $k = -0.5$}{86}%
    \metodo{sonnet}{niblack}{20}{$r = 20$ e $k = -0.5$}{84}

    \caption{Exemplos de aplicação do método \texttt{niblack}.}
    \label{fig:niblack}
\end{figure}
\subsection{Método de Sauvola e Pietikäinen}

Assim como no método de Niblack, esse método se baseia na média e desvio padrão dos pixels da vizinhança, definindo o limiar como $T(x, y) = \mu(x, y) \left[1 + k \left(\frac{\sigma(x, y)}{R} - 1\right)\right]$. Essa fórmula faz o método funcione melhor com os anteriores em regiões de iluminação irregular, já que a baixa variância da região mantém o limiar da figura baixo.

Por exemplo, a \cref{fig:sauvola:sonnet} consegue reduzir ainda mais o ruído, mas ao mesmo tempo manter mais detalhes do que na \cref{fig:niblack}. Isso era esperado, já que o método foi desenvolvido principalmente para o tratamento de imagens textuais.

Apesar disso, o método ainda não consegue separar objetos em imagens rica em detalhes, como a \texttt{monarch.pgm} (\cref{fig:sauvola:monarch}). Em imagens como a \texttt{retina.pgm}, ele consegue remover as manchas que existiam na \cref{fig:global}, mas aumenta o ruído na imagem.

\begin{figure}[H]
    \centering
    \metodo{sonnet}{sauvola}{}{$k = 0.08$ e $R = 100$}{93}%
    \metodo{retina}{sauvola}{}{$k = -0.5$ e $R = 100$}{7}\\[8pt]
    \metodo{wedge}{sauvola}{}{$k = 0.08$ e $R = 200$}{98}%
    \metodo{monarch}{sauvola}{}{$k = 0.5$ e $R = 300$}{94}

    \caption{Exemplos de aplicação do método \texttt{sauvola} com $r = 10$.}
    \label{fig:sauvola}
\end{figure}
\subsection{Método de Phansalkar, More e Sabale}

O método desenvolvido por Phansalkar et al teve forte influência do método de Sauvola, apresentado na seção anterior. A principal diferença é um fator exponencial na fórmula, o que faz com que o limiar não caia muito rapidamente em regiões de baixa intensidade. Essa técnica foi pensanda com foco em imagens celulares, com tratamento para cores e usando a intensidade normalizada em $[0, 1]$. Nessa técnica, a fórmula do limiar é dada por:
\[
    T(x, y) = \mu(x, y) \left[1 + p e^{-q \mu(x, y)} + k \left(\frac{\sigma(x, y)}{R} - 1\right)\right]
\]

\begin{figure}[H]
    \centering
    \metodo[wide]{retina}{phansalkar}{}{$r = 5$, $k = 0.2$, $R = 4$, $p = 0.2$ e $q = 0.5$}{5}\\[8pt]
    \metodo{monarch}{phansalkar}{}{$r = 10$, $k = 0.9$, $R = 0.7$, $p = 5$ e $q = 10$}{95}%
    \metodo{wedge}{phansalkar}{}{$r = 10$, $k = 0.1$, $R = 0.9$, $p = 12$ e $q = 16$}{83}

    \caption{Exemplos de aplicação do método \texttt{phansalkar}.}
    \label{fig:phansalkar}
\end{figure}

Podemos ver que o método consegue reduzir a força da borda da retina na \cref{fig:phansalkar:retina}, mas mantendo os vasos sanguíneos. Além disso, na \texttt{monarch.pgm}, a borboleta consegue ser mantida removendo ainda boa parte da folhagem, mas alguns detalhes são perdido. Esse método consegue também começar a marcar o centro do objeto na \texttt{wedge.pgm} (\cref{fig:phansalkar:wedge}).
\subsection{Método do Contraste}

\begin{figure}[H]
    \centering
    \metodo{baboon}{contraste}{}{$r = 50$}{42}%
    \metodo{sonnet}{contraste}{}{$r = 20$}{26}

    \caption{Exemplos de aplicação do método \texttt{contraste}.}
    \label{fig:contraste}
\end{figure}

Esse método define objeto como pixels mais próximos do mínimo local do que do máximo, deixando o restante como fundo. Na prática, ele acaba funcionando como o método de Bernsen com o resultado invertido, por isso, o que foi discutido na \cref{sec:bernsen} também vale aqui.

No entanto, note que as porcentagens do \texttt{sonnet.pgm} (figuras \ref{fig:bernsen:sonnet} e \ref{fig:contraste:sonnet}) não soman $100\%$. Isso se deve aos pixels que se encontram exatamente na média da sua vizinhança e acaba sumindo em raios maiores, como da \cref{fig:contraste:baboon}.

\subsection{Método da Média}

O método da média define o limiar apenas como a média da vizinhança, isto é, $T(x, y) = \mu(x, y)$. Isso funciona razoavelmente bem para imagens onde o objeto tem apenas uma cor, como é caso das figuras \ref{fig:media:fiducial} e \ref{fig:media:wedge}.

No entanto, imagens pequenas como \texttt{retina.pgm}, onde o raio não pode ser muito grande, o método acaba deixando resultados pouco interessantes.

\begin{figure}[H]
    \centering
    \metodo{fiducial}{media}{}{$r = 50$}{64}%
    \metodo{wedge}{media}{}{$r = 50$}{52}\\[8pt]
    \metodo[wide]{retina}{media}{}{$r = 30$}{43}

    \caption{Exemplos de aplicação do método \texttt{media}.}
    \label{fig:media}
\end{figure}

\section{Método da Mediana}

O último método seleciona o limiar como a mediana da vizinhança. Isso faz com que a probabilidade do pixel de ser maior ou menor que o limiar fique em torno de $50\%$, gerando muito ruído na imagem. Podemos ver esse efeito em objetos simples, de apenas um cor, como na \cref{fig:mediana:wedge}.

Em objetos detalhados (\cref{fig:mediana:peppers}), a técina acaba mantendo o aspecto da figura original, em vez de enfatizar os objetos da figura.

\begin{figure}[H]
    \centering
    \metodo{wedge}{mediana}{}{$r = 50$}{42}%
    \metodo{peppers}{mediana}{}{$r = 60$}{50}

    \caption{Exemplos de aplicação do método \texttt{mediana}.}
    \label{fig:mediana}
\end{figure}



    \section{Conclusão}

        Podemos ver que o método de Phansalkar et al., apesar de ter mais parâmetros, também é a técnica com mais controle do resultado. Além disso, com os parâmetros corretos, esse método pode funcionar exatamente como as outras técnicas baseadas na média da vizinhança.

        Apesar disso, para objetos mais complexos, com grande variações de intensidade dentro do próprio, mas com boa iluminação em todo o objeto, o método de Bernsen e o método do contraste podem funcionar bem. O método global, também serve em caso específicos.

\end{document}
